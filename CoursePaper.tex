\documentclass{article}
\usepackage[utf8]{inputenc}
\usepackage[russian]{babel}
\usepackage{setspace,amsmath}
\usepackage{lipsum}
\usepackage[usestackEOL]{stackengine}
\usepackage{lipsum}
\usepackage{kantlipsum}
\usepackage[left=2cm,right=2cm,top=2cm,bottom=2cm,bindingoffset=0cm]{geometry}
\newcommand\zz[1]{\par{\normalsize\strut #1} \hfill\ignorespaces}
\addto\captionsrussian{\def\refname{Список использованных источников}}
\begin{document}
 
\begin{center}
\textbf{
ПРАВИТЕЛЬСТВО РОССИЙСКОЙ ФЕДЕРАЦИИ\\
НАЦИОНАЛЬНЫЙ ИССЛЕДОВАТЕЛЬСКИЙ УНИВЕРСИТЕТ\\
«ВЫСШАЯ ШКОЛА ЭКОНОМИКИ»\\
Факультет компьютерных наук
Образовательная программа «Программная инженерия»}\\
\end{center}
УДК 004.852\\
~\\
\zz{СОГЛАСНОВАНО}УТВЕРЖДАЮ
\zz{Руководитель,}Академический руководитель
\zz{доцент департамента}образовательной программы
\zz{программной инженерии}«Программная инженерия»
\zz{факультета компьютерных наук,}профессор департамента программной
\zz{канд. техн. наук}инженерии, канд. техн. наук
\zz{\noindent\rule{3cm}{0.4pt} И.В. Иванов}\noindent\rule{3cm}{0.4pt} В.В. Шилов
\zz{«\noindent\rule{1cm}{0.4pt}»\noindent\rule{2cm}{0.4pt}20\noindent\rule{0.5cm}{0.4pt}г.}«\noindent\rule{1cm}{0.4pt}»\noindent\rule{2cm}{0.4pt}20\noindent\rule{0.5cm}{0.4pt}г.
\begin{center}
\topskip=0pt
\vspace*{\fill}
\textbf{Отчет\\
по исследовательскому курсовому проекту}\\
на тему «Обучение с подкреплением для задач распределения ресурсов в облаке»\\
по направлению подготовки бакалавров 09.03.04 «Программная инженерия»\\
\vspace*{\fill}
\end{center}
{\raggedleft\vfill\Longstack[l]{%
  Выполнил:\\
  Студент группы БПИ204\\
  образовательной программы\\
  09.03.04 «Программная инженерия»\\
  Пеганов Никита Сергеевич\\
  ~\\
  \noindent\rule{6cm}{0.4pt}\\
}\par
}
\begin{center}
\vspace*{\fill}{
  Москва 2021}
\end{center}
\newpage
\begin{center}
\section {Реферат}
\end{center}
\textbf{Перечень ключевых слов: }обучение с подкреплением, reinforcement learning, RL, распределение ресурсов в облаке, облачные технологии, облачные ресурсы.\\
\textbf{Краткое описание объекта исследования:} особенности выделения ресурсов при работе с облачными сервисами.\\
\textbf{Краткое описание предмета исследования:} применимость обучения с подкреплением для задачи распределения ресурсов в облаке.\\
\textbf{Цель проекта:} исследование применимости обучения с подкреплением в задачах распределения облачных ресурсов. Сравнение данного подхода с другими методами решения задачи. \\
\textbf{Метод или методология проведения работы:} метод эксперимента.\\
\textbf{Результаты проекта:}\\
\textbf{Апробация результатов:}\\
\newpage
\begin{center}
\section {Содержание}
\tableofcontents
\end{center}
\newpage
\section {Основные термины, определения и сокращения}
IT (произносится ай-ти, сокращение от англ. Information Technology) — информационные технологии\\
RL (англ. reinforcement learning) — обучение с подкреплением\\
\newpage
\begin{center}
\section {Введение}
\end{center}
\textbf{Актуальность}\\
Облачные технологии позволяют обеспечить круглосуточную и бесперебойную работу интернет-сервисов, что делает их востребованными во всех сферах IT-индустрии. Облачными вычислениями занимаются Amazon, Google, Huawei и другие крупнейшие информационные компании\cite{litlink1}\cite{litlink2}. В 2020 году мировой рынок облачных вычислений оценивается в 289.25 миллиардов долларов\cite{litlink3}. Распределение облачных ресурсов — одна из важнейших задач облачных вычислений.\\
\textbf{Предмет исследования}\\
Возможность использования обучения с подкреплением для решения задачи распределения ресурсов облака.
\textbf{Методы исследования}\\
Экспериментальное сравнение показателей RL в ходе решения задачи распределения облачных ресурсов с иными используемыми на практике способами. Для наглядности в работе также решена близкая задача: автоматическая игра в "Тетрис" с помощью обучения с подкреплением.\\
\textbf{Цели и задачи работы}\\
Определение эффективность обучения с подкреплением в задаче распределения ресурсов в облаке.\\
\textbf{Новизна и достоверность полученных результатов}\\
\textbf{Теоретическая значимость}\\
\textbf{Практическая ценность}\\
В случае превосходства RL над другими методами в рамках решения задачи распределения облачных ресурсов применение данного способа машинного обучения способно сократить нагрузку на сервера, предоставляющие доступ к облачным сервисам. Это позволит уменьшить расходы компаний на поддержку их работоспособности, а также расходы на производство при сокращении количества серверов. Проект имеет практическую ценность для экологии: уменьшение расходов электроэнергии приведет к уменьшению углеродного следа компаниий.	\\
\newpage
\begin{center}
\section {Обзор и анализ источников}
\end{center}
Первая часть курсовой работы посвящена автоматической игре в "Тетрис" с помощью обучения с подкреплением. Рассмотрим исследования данной задачи и ее решения. В статье "Tetris is Hard, Even to Approximate"\cite{litlink4} доказывается, что игра Тетрис является NP-полной задачей. Это одна из причин схожести данной игры с распределением ресурсов в облаке\cite{litlink5}. В статье Playing the Original Game Boy Tetris Using a Real Coded Genetic Algorithm\cite{litlink6} используется генетический алгоритм для симуляции игры в тетрис. В данной работе метриками успеха автор считает максимальное число удаленных строк до поражения и среднее число удаленных строк у запущенного несколько раз алгоритма. Обе метрики значительно уступают роевым оптимизациям,  продемонстрированным в работах Apply ant colony optimization to tetris\cite{litlink7} и Swarm tetris: Applying particle swarm optimization to tetris\cite{litlink8}. Примером использования RL для игры в Тетрис является статья A deep reinforcement learning bot that plays tetris\cite{litlink9}.
\newpage
\begin{center}
\section {Выбор методов, алгоритмов, моделей для решения поставленных задач}
\end{center}
\newpage
\begin{center}
\section {Описание выбранных или предлагаемых методов, алгоритмов, моделей, методик}
\end{center}
\newpage
\begin{center}
\section {Описание эксперимента}
\end{center}
\newpage
\begin{center}
\section {Анализ и оценка полученных результатов}
\end{center}
\newpage
\begin{center}
\section {Заключение}
\end{center}
\newpage
\begin{center}
\section {Перспективы дальнейших исследований по данной тематике}
\end{center}
\newpage
\begin{center}
\addcontentsline{toc}{section}{Список использованных источников}
\begin{thebibliography}{}
    \bibitem{litlink1}  \textit{Arif Mohamed} (2018) A history of cloud computing // Сайт Сomputerweekly.com. 9 апреля (https://www.computerweekly.com/feature/A-history-of-cloud-computing) Просмотрено: 11.12.2021.
    \bibitem{litlink2}  \textit{Matt Kapko} (2021) Can Huawei ‘Reinvent Itself’ as a Cloud Leader? // Сайт Sdxcentral.com. 26 апреля (https://www.sdxcentral.com/articles/news/can-huawei-reinvent-itself-as-a-cloud-leader/2021/04/) Просмотрено: 11.12.2021
    \bibitem{litlink3} \textit{Laura Wood} (2021) Global Cloud Computing Market (2020 to 2026) - by Service, Deployment, Application Type, End-user and Region Businesswire.com. 24 августа (https://www.businesswire.com/news/home/20210824005585/en/Global-Cloud-Computing-Market-2020-to-2026---by-Service-Deployment-Application-Type-End-user-and-Region---ResearchAndMarkets.com) Просмотрено: 11.12.2021
\bibitem{litlink4}  \textit{Erik D. Demaine, Susan Hohenberger, David Liben-Nowell} (2002) Tetris is Hard, Even to Approximate // Сайт Arxiv.org. 21 октября (https://arxiv.org/abs/cs/0210020) Просмотрено: 11.12.2021
\bibitem{litlink5}  \textit{Harvinder Singh, Anshu Bhasin, Parag Ravikant Kaveri} (2021) QRAS: efficient resource allocation for task scheduling in cloud computing // Сайт Researchgate.net. Апрель (https://www.researchgate.net/publication/350192028\_QRAS\_efficient\_resource\_allocation\_for\_task\_\\scheduling\_in\_cloud\_computing) Просмотрено: 11.12.2021
\bibitem{litlink6} \textit{Renan Samuel da Silva, Rafael Stubs Parpinelli} (2017) Playing the Original Game Boy Tetris Using a Real Coded Genetic Algorithm // Сайт Researchgate.net. Октрябрь (https://www.researchgate.net/publication/322321608\_Playing\_the\_Original\_Game\_Boy\_Tetris\_Using\_a\_\\Real\_Coded\_Genetic\_Algorithm) Просмотрено: 11.12.2021
\bibitem{litlink7} \textit{X. Chen, H. Wang, W. Wang, Y. Shi, and Y. Gao} (2009) Apply ant colony optimization to tetris // Сайт Dl.acm.org. 8 июля (https://dl.acm.org/doi/10.1145/1569901.1570136) Просмотрено: 11.12.2021
\bibitem{litlink8} \textit{L. Langenhoven, W. S. van Heerden, and A. P. Engelbrecht} (2010) Swarm tetris: Applying particle swarm optimization to tetris // Сайт Ieeexplore.ieee.org. 18-23 июля (https://ieeexplore.ieee.org/document/5586033) Просмотрено: 11.12.2021
\bibitem{litlink9} \textit{nuno-faria, nlinker (Nick Linker)} (2019) A bot that plays tetris using deep reinforcement learning. // Сайт Github.com. 7 сентября (https://github.com/nuno-faria/tetris-ai) Просмотрено: 11.12.2021
\end{thebibliography}
\end{center}
\newpage
\begin{center}
\section {Приложения}
\end{center}
\zz{}\textbf{Приложение 1\\}
Ссылка на репозиторий проекта с исходным кодом и всеми использованными материалами.\\
https://github.com/NikPeg/Reinforcement-learning-for-resource-allocation-tasks-in-the-cloud
\end{document}