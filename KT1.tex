\documentclass{article}
\usepackage[utf8]{inputenc}
\usepackage[russian]{babel}
\usepackage{setspace,amsmath}
\usepackage{lipsum}
\usepackage[usestackEOL]{stackengine}
\usepackage{lipsum}
\usepackage{kantlipsum}
\usepackage[left=2cm,right=2cm,top=2cm,bottom=2cm,bindingoffset=0cm]{geometry}
\usepackage[pdftex]{graphicx}
\graphicspath{{pictures/}}
\DeclareGraphicsExtensions{.pdf,.png,.jpg}
\newcommand\zz[1]{\par{\normalsize\strut #1} \hfill\ignorespaces}
\addto\captionsrussian{\def\refname{Список использованных источников}}
\begin{document}
\newpage
\begin{center}
\section {Реферат}
\end{center}
\textbf{Перечень ключевых слов: }обучение с подкреплением, reinforcement learning, RL, распределение ресурсов в облаке, облачные технологии, облачные ресурсы, Tetris, OpenAI Gym, TensorFlow, KerasRL.\\
\textbf{Краткое описание объекта исследования:} особенности выделения ресурсов при работе с облачными сервисами.\\
\textbf{Краткое описание предмета исследования:} применимость обучения с подкреплением для задачи распределения ресурсов в облаке.\\
\textbf{Цель проекта:} исследование применимости обучения с подкреплением в задачах распределения облачных ресурсов. Сравнение данного подхода с другими методами решения задачи. \\
\textbf{Метод или методология проведения работы:} метод эксперимента.\\
\textbf{Результаты проекта:} выяснение границ применимости обучения с подкреплением для решения задач распределения облачных ресурсов.\\
\textbf{Апробация результатов:}\\
\newpage
\begin{center}
\section {Содержание}
\tableofcontents
\end{center}
\newpage
\section {Основные термины, определения и сокращения}
IT (произносится ай-ти, сокращение от англ. Information Technology) — информационные технологии.\\
RL (англ. reinforcement learning) — обучение с подкреплением.\\
CPU (англ. central processing unit) — электронное устройство, исполняющее машинный код программ, главная часть аппаратного обеспечения компьютера. Иногда также называется микропроцессором или процессором.\\
RAM (англ. Random Access Memory) — запоминающее устройство с произвольным доступом, один из видов памяти компьютера, позволяющий единовременно получить доступ к любой ячейке по её адресу на чтение или запись.
\newpage
\begin{center}
\section {Введение}
\end{center}
В первой части работы описано применение обучения с подкреплением для визуализации компьютерной игры "Тетрис"\cite{litlink8}. Эта игра представляет собой клетчатое поле шириной 10 клеток и высотой 20 клеток. В верхней части поля друг за другом появляются клетчатые фигурки, состоящие из 4 клеток (тетрамино). Фигурки имеют форму, напоминающую форму букв "I", "Z", "L", "T", а также квадрат из четырех клеток. Пользователь имеет возможность поворачивать фигурку на 90°, а также двигать ее по горизонтали во время падения. В случае заполнения одной из строк частями фигурок строка "исчезает": все фигурки выше нее опускаются на одну строку вниз. Каждая "исчезнувшая" строка приносит игроку 1 очко.\\
Во второй части работы обучение с подкреплением применено для решения задач распределения облачных ресурсов.\\
\textbf{Актуальность}\\
Облачные технологии позволяют обеспечить круглосуточную и бесперебойную работу интернет-сервисов, что делает их востребованными во всех сферах IT-индустрии. Облачными вычислениями занимаются Amazon, Google, Huawei и другие крупнейшие информационные компании\cite{litlink2}\cite{litlink12}. В 2020 году мировой рынок облачных вычислений оценивается в 289.25 миллиардов долларов\cite{litlink11}. Распределение облачных ресурсов — одна из важнейших задач облачных вычислений.\\
\textbf{Предмет исследования}\\
Возможность использования обучения с подкреплением для решения задачи распределения ресурсов облака.
\textbf{Методы исследования}\\
Экспериментальное сравнение показателей RL в ходе решения задачи распределения облачных ресурсов с иными используемыми на практике способами. Для наглядности в работе также решена близкая задача: автоматическая игра в "Тетрис" с помощью обучения с подкреплением.  Данная компьютерная игра выбрана неслучайно: она имеет концепции, сходные с основной задачей. Во-первых, ее основная цель — упаковка фигур. В решаемой задаче так же требуется распределять задачи пользователей между имеющимися ресурсами серверов. Во-вторых, игра имеет два параметра — координаты X и Y. Основная задача так же имеет два параметра, которые требуется распределять: CPU и RAM. Также решение задачи автоматической игры в "Тетрис" позволила научиться применять использованные библиотеки и фреймворки на практике.\\
\textbf{Цели и задачи работы}\\
Определение эффективность обучения с подкреплением в задаче распределения ресурсов в облаке.\\
\textbf{Новизна и достоверность полученных результатов}\\
\textbf{Теоретическая значимость}\\
\textbf{Практическая ценность}\\
В случае превосходства RL над другими методами в рамках решения задачи распределения облачных ресурсов применение данного способа машинного обучения способно сократить нагрузку на сервера, предоставляющие доступ к облачным сервисам. Это позволит уменьшить расходы компаний на поддержку их работоспособности, а также расходы на производство при сокращении количества серверов. Проект имеет практическую ценность для экологии: уменьшение расходов электроэнергии приведет к уменьшению углеродного следа компаниий.	\\
\newpage
\begin{center}
\section {Обзор и анализ источников}
\end{center}
Первая часть курсовой работы посвящена автоматической игре в "Тетрис" с помощью обучения с подкреплением. Рассмотрим исследования данной задачи и ее решения. В статье "Tetris is Hard, Even to Approximate"\cite{litlink6} доказывается, что игра Тетрис является NP-полной задачей. Это одна из причин схожести данной игры с распределением ресурсов в облаке\cite{litlink6}. В статье Playing the Original Game Boy Tetris Using a Real Coded Genetic Algorithm\cite{litlink7} используется генетический алгоритм для симуляции игры в тетрис. В данной работе метриками успеха автор считает максимальное число удаленных строк до поражения и среднее число удаленных строк у запущенного несколько раз алгоритма. Обе метрики значительно уступают роевым оптимизациям,  продемонстрированным в работах Apply ant colony optimization to tetris\cite{litlink19} и Swarm tetris: Applying particle swarm optimization to tetris\cite{litlink10}. Примером использования RL для игры в Тетрис является статья A deep reinforcement learning bot that plays tetris\cite{litlink13}.
\newpage
\begin{center}
\section {Выбор методов, алгоритмов, моделей для решения поставленных задач}
\end{center}
Для демонстрации работы обучения с подкреплением на примере игры "Тетрис" требовалось выбрать среду для симуляции игры, а также библиотеку для реализации машинного обучения.\\
В качестве среды был рассмотрен симулятор устройства для игр "Game Boy"\ PyBoy\cite{litlink3}. Однако он был отвергнут в пользу более популярной и более простой в использовании библиотеки gym-tetris\cite{litlink5}, являющейся частью OpenAI Gym\cite{litlink14} — среды для симуляции известных компьютерных игр и физических задач.\\
При выборе библиотеки были рассмотрены pyqlearning\cite{litlink1} и Tensorforce\cite{litlink17}. Однако выбрана была библиотека KerasRL\cite{litlink9}, надстройка над фреймворком TensorFlow\cite{litlink16}. Выбор был сделан в пользу KerasRL из-за совместимости со средой OpenAI Gym.\\
В игре "Тетрис" могут быть использованы различные метрики для рассчета награды агента. Например, число убранных строк, количество сброшенных фигурок, число ходов до проигрыша, переход на новую скорость и другие. Для простоты в качестве награды было выбрано число убранных строк.\\
\newpage
\begin{center}
\addcontentsline{toc}{section}{Список использованных источников}
\begin{thebibliography}{}
\bibitem{litlink1} accel-brain, chimera0 [Электронный ресурс] / Pypi. Режим доступа: https://pypi.org/project/pyqlearning/, свободный. (дата обращения: 01.02.2022)

\bibitem{litlink2} Arif Mohamed. A history of cloud computing [Электронный ресурс]: Сomputerweekly, 2018 – Режим доступа: https://www.computerweekly.com/feature/A-history-of-cloud-computing, свободный. (дата обращения: 11.12.2021)

\bibitem{litlink3} Baekalfen (Mads Ynddal) [Электронный ресурс] / Github. Режим доступа: https://github.com/Baekalfen/PyBoy, свободный. (дата обращения: 01.02.2022)

\bibitem{litlink4} Cade Metz. TensorFlow, Google's Open Source AI, Signals Big Changes in Hardware Too. [Электронный ресурс]: Wired, 2015 – Режим доступа: https://www.wired.com/2015/11/googles-open-source-ai-tensorflow-signals-fast-changing-hardware-world, свободный. (дата обращения: 02.02.2022) 

\bibitem{litlink5} Christian Kauten [Электронный ресурс] / Pypi. Режим доступа: https://github.com/Baekalfen/PyBoy, свободный. (дата обращения: 01.02.2022)

\bibitem{litlink6} Erik D. Demaine. Tetris is Hard, Even to Approximate / Erik D. Demaine, Susan Hohenberger, David Liben-Nowell // [Электронный ресурс]: Arxiv, 2002 – Режим доступа: https://arxiv.org/abs/cs/0210020, свободный. (дата обращения: 11.12.2021)

\bibitem{litlink7} Harvinder Singh. QRAS: efficient resource allocation for task scheduling in cloud computing / Harvinder Singh, Anshu Bhasin, Parag Ravikant Kaveri // [Электронный ресурс]: Researchgate, 2021 – Режим доступа: https://www.researchgate.net/publication/350192028\_QRAS\_efficient\_resource\_allocation\_for\_task\_\\scheduling\_in\_cloud\_computing, свободный. (дата обращения: 11.12.2021)

\bibitem{litlink8} Kent, Steven. The Ultimate History of Video Games: From Pong to Pokemon: The Story Behind the Craze That Touched Our Lives and Changed the World (1st ed.). – New York, USA: Three Rivers Press, 2001. – С. 377-381.

\bibitem{litlink9} Keras-RL [Электронный ресурс] / Github. Режим доступа: https://github.com/keras-rl/keras-rl, свободный. (дата обращения: 01.02.2022)

\bibitem{litlink10} L. Langenhoven. Swarm tetris: Applying particle swarm optimization to tetris / L. Langenhoven, W. S. van Heerden, and A. P. Engelbrecht // [Электронный ресурс]: Ieeexplore, 2010 – Режим доступа: https://ieeexplore.ieee.org/document/5586033, свободный. (дата обращения: 11.12.2021)

\bibitem{litlink11} Laura Wood. Global Cloud Computing Market (2020 to 2026) - by Service, Deployment, Application Type, End-user and Region [Электронный ресурс]: Businesswire, 2021 – Режим доступа: https://www.businesswire.com/news/home/20210824005585/en/Global-Cloud-Computing-Market-2020-to-2026---by-Service-Deployment-Application-Type-End-user-and-Region---ResearchAndMarkets.com, свободный. (дата обращения: 11.12.2021)

\bibitem{litlink12} Matt Kapko. Can Huawei ‘Reinvent Itself’ as a Cloud Leader? [Электронный ресурс]: Sdxcentral, 2021 – Режим доступа: https://www.sdxcentral.com/articles/news/can-huawei-reinvent-itself-as-a-cloud-leader/2021/04, свободный. (дата обращения: 11.12.2021)

\bibitem{litlink13} nuno-faria, nlinker (Nick Linker) [Электронный ресурс] / Github. Режим доступа: https://github.com/nuno-faria/tetris-ai, свободный. (дата обращения: 11.12.2021)

\bibitem{litlink14} OpenAI [Электронный ресурс] / Github. Режим доступа: https://github.com/openai/gym, свободный. (дата обращения: 01.02.2022)

\bibitem{litlink15} Renan Samuel da Silva. Playing the Original Game Boy Tetris Using a Real Coded Genetic Algorithm / Renan Samuel da Silva, Rafael Stubs Parpinelli // [Электронный ресурс]: Researchgate, 2017 – Режим доступа: https://www.researchgate.net/publication/322321608\_Playing\_the\_Original\_Game\_Boy\_Tetris\_Using\_a\_\\Real\_Coded\_Genetic\_Algorithm, свободный. (дата обращения: 11.12.2021)

\bibitem{litlink16} tensorflow [Электронный ресурс] / Github. Режим доступа: https://github.com/tensorflow/tensorflow, свободный. (дата обращения: 01.02.2022)

\bibitem{litlink17} Tensorforce [Электронный ресурс] / Github. Режим доступа: https://github.com/tensorforce/tensorforce, свободный. (дата обращения: 01.02.2022)

\bibitem{litlink18} Vihar Kurama. Обучение с подкреплением на языке Python / Vihar Kurama, Samhita Alla // [Электронный ресурс]: Habr, 2018 – Режим доступа: https://habr.com/ru/company/piter/blog/434738, свободный. (дата обращения: 02.02.2022)

\bibitem{litlink19} X. Chen. Apply ant colony optimization to tetris / X. Chen, H. Wang, W. Wang, Y. Shi, and Y. Gao // [Электронный ресурс]: Dl, 2009 – Режим доступа: https://dl.acm.org/doi/10.1145/1569901.1570136, свободный. (дата обращения: 11.12.2021)

\end{thebibliography}
\end{center}
\newpage
\begin{center}
\section {Приложения}
\end{center}
\zz{}\textbf{Приложение 1\\}
Ссылка на репозиторий проекта с исходным кодом и всеми использованными материалами.\\
https://github.com/NikPeg/Reinforcement-learning-for-resource-allocation-tasks-in-the-cloud
\zz{}\textbf{Приложение 2\\}
Календарный план работ:\\
1. Завершить работу с тетрисом (31 января)\\
2. Изучение материалов по применению RL в облачных задачах, их описание в тексте курсовой работы, выбор конкретной практической задачи для решения, планирование эксперимента (28 февраля)\\
3. Реализация задачи в коде (31 марта)\\
4. Усложнение задачи: изменение условий среды или применение другой метрики (30 апреля)\\
5. Исправление ошибок в работе, оформление (20 мая)\\
\end{document}